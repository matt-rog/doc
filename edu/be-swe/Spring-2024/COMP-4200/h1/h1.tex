\documentclass[12pt]{article}
\usepackage[margin=1in]{geometry}
\usepackage[all]{xy}

\usepackage{amsmath,amsthm,amssymb,color,latexsym}
\usepackage{geometry}  
\geometry{letterpaper}  
\usepackage{graphicx}

\newtheorem{problem}{Problem}

\newenvironment{solution}[1][\it{Solution}]{\textbf{#1. } }{$\square$}

\begin{document}
\noindent COMP 4200-001 Spring 2024\hfill Homework 1\\
Matthew Rogers \hfill1/26

\begin{problem}\end{problem}
0.1f: In the set constructor, the conditional $n=n+1$ is a contradiction, which always evaluates to false. The set is never populated and thus is $\emptyset$.\\

0.1d: The set will contain palindromes, as each string $s=\text{reverse}(s)$, with each string being defined over the binary alphabet.\\

0.6d: The domain of $g$ is the cross product $X\times Y$ where $Y$ is the range of $f$. The range of $g$ is also $Y$.\\

0.6e: $g(4,f(4))=8$

\begin{problem}\end{problem}
\begin{enumerate}
\item{
Base: $n=1$
$$\begin{aligned}
            5^1+5&<5^{1+1}\\
        	10&<5^2\\
        	10&<25
            \end{aligned}$$
Thus, the base case holds\\
Induction Hypothesis: $n=k\implies5^{k}+5<5^{k+1}$ when $k\geq1$
$$\begin{aligned}
            5^{k+1}+5&<5^{k+1+1}\\
            5^k+5&<5^{k+1}\cdot5\\
            5(5^k+1)&<5(5^{k+1})\\
            5^k+1&<5^{k+1}\\
            \text{IS:}\quad5^k+5&<5^{k+1}\\
            &\qquad\qquad\square
            \end{aligned}$$
}
\item{
Base: $n=1$
$$\begin{aligned}
            \sum_{i=1}^n(-1)^ii^2&=(-1)^n\frac{n(n+1)}{2}\\
            \sum_{i=1}^1(-1)^ii^2&=(-1)^1\frac{1(1+1)}{2}\\-1&=-1\\
            \end{aligned}$$
Thus, the base case holds\\
Induction Hypothesis: $n=k\implies\sum_{i=1}^k(-1)^ii^2=(-1)^k\frac{k(k+1)}{2}$ when $k\geq1$
$$\begin{aligned}
            \sum_{i=1}^{k+1}(-1)^ii^2&=(-1)^{k+1}\frac{(k+1)(k+1+1)}{2}\\
            \sum_{i=1}^{k}(-1)^ii^2+(-1)^{k+1}(k+1)^2&=(-1)^{k+1}\frac{(k+1)(k+1+1)}{2}\\
            \text{IS:}\quad(-1)^k\frac{k(k+1)}{2}+(-1)^{k+1}(k+1)^2&=(-1)^{k+1}\frac{(k+1)(k+1+1)}{2}\\
            (-1)^k\frac{k^2+k}{2}+(-1)^{k+1}(k^2+2k+1)&=(-1)^{k+1}\frac{k^2+3k+2}{2}\\
            (-1)^k(\frac{k^2}{2}+\frac{k}{2})+(-1)^{k+1}(k^2+2k+1)&=(-1)^{k+1}(\frac{k^2}{2}+\frac{3k}{2}+1)\\
            (-1)^k(\frac{k^2}{2}+\frac{k}{2})+(-1)^{k}(-k^2-2k)&=(-1)^{k}(\frac{-k^2}{2}+\frac{-3k}{2})\\
            (-1)^k(\frac{k^2}{2}+\frac{k}{2}+-k^2-2k)&=(-1)^{k}(\frac{-k^2}{2}+\frac{-3k}{2})\\
            (-1)^{k}(\frac{-k^2}{2}+\frac{-3k}{2})&=(-1)^{k}(\frac{-k^2}{2}+\frac{-3k}{2})\\
            &\qquad\qquad\qquad\qquad\qquad\square
            \end{aligned}$$
}
\end{enumerate}

\end{document}
