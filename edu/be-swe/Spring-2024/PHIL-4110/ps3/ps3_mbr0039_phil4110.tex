\documentclass{article}

\usepackage{fancyhdr}
\usepackage{extramarks}
\usepackage{amsmath}
\usepackage{amsthm}
\usepackage{amsfonts}
\usepackage{tikz}
\usepackage[plain]{algorithm}
\usepackage{algpseudocode}
\usepackage{physics}
\usepackage{changepage}

\usetikzlibrary{automata,positioning}

%
% Basic Document Settings
%

\topmargin=-0.45in
\evensidemargin=0in
\oddsidemargin=0in
\textwidth=6.5in
\textheight=9.0in
\headsep=0.25in

\linespread{1.1}

\pagestyle{fancy}
\lhead{\hmwkAuthorName}
\chead{\hmwkClass\ (\hmwkClassInstructor\ \hmwkClassTime): \hmwkTitle}
\rhead{\firstxmark}
\lfoot{\lastxmark}
\cfoot{\thepage}

\renewcommand\headrulewidth{0.4pt}
\renewcommand\footrulewidth{0.4pt}

\setlength\parindent{0pt}

%
% Create Problem Sections
%

\newcommand{\enterProblemHeader}[1]{
    \nobreak\extramarks{}{Problem \arabic{#1} continued on next page\ldots}\nobreak{}
    \nobreak\extramarks{Problem \arabic{#1} (continued)}{Problem \arabic{#1} continued on next page\ldots}\nobreak{}
}

\newcommand{\exitProblemHeader}[1]{
    \nobreak\extramarks{Problem \arabic{#1} (continued)}{Problem \arabic{#1} continued on next page\ldots}\nobreak{}
    \stepcounter{#1}
    \nobreak\extramarks{Problem \arabic{#1}}{}\nobreak{}
}

\setcounter{secnumdepth}{0}
\newcounter{partCounter}
\newcounter{homeworkProblemCounter}
\setcounter{homeworkProblemCounter}{1}
\nobreak\extramarks{Problem \arabic{homeworkProblemCounter}}{}\nobreak{}

%
% Homework Problem Environment
%
% This environment takes an optional argument. When given, it will adjust the
% problem counter. This is useful for when the problems given for your
% assignment aren't sequential. See the last 3 problems of this template for an
% example.
%
\newenvironment{homeworkProblem}[1][-1]{
    \ifnum#1>0
        \setcounter{homeworkProblemCounter}{#1}
    \fi
    \section{Problem \arabic{homeworkProblemCounter}}
    \setcounter{partCounter}{1}
    \enterProblemHeader{homeworkProblemCounter}
}{
    \exitProblemHeader{homeworkProblemCounter}
}

\newenvironment{ind}
  {\adjustwidth{3em}{0pt}}
  {\endadjustwidth}

%
% Homework Details
%   - Title
%   - Due date
%   - Class
%   - Section/Time
%   - Instructor
%   - Author
%

\newcommand{\hmwkTitle}{Problem Set 3}
\newcommand{\hmwkDueDate}{February 26, 2024}
\newcommand{\hmwkClass}{PHIL 4110}
\newcommand{\hmwkClassTime}{}
\newcommand{\hmwkClassInstructor}{Professor Lockhart}
\newcommand{\hmwkAuthorName}{\textbf{Matthew Rogers}}

%
% Title Page
%

\title{
    \vspace{2in}
    \textmd{\textbf{\hmwkClass:\ \hmwkTitle}}\\
    \normalsize\vspace{0.1in}\small{Due\ on\ \hmwkDueDate}\\
    \vspace{0.1in}\large{\textit{\hmwkClassInstructor}}
    \vspace{3in}
}

\author{\hmwkAuthorName}
\date{}

\renewcommand{\part}[1]{\textbf{\large Part \Alph{partCounter}}\stepcounter{partCounter}\\}

%
% Various Helper Commands
%

% Useful for algorithms
\newcommand{\alg}[1]{\textsc{\bfseries \footnotesize #1}}

% For derivatives
\newcommand{\deriv}[1]{\frac{\mathrm{d}}{\mathrm{d}x} (#1)}

% For partial derivatives
\newcommand{\pderiv}[2]{\frac{\partial}{\partial #1} (#2)}

% Integral dx
\newcommand{\dx}{\mathrm{d}x}

% Alias for the Solution section header
\newcommand{\solution}{\textbf{\large Solution}}

% Probability commands: Expectation, Variance, Covariance, Bias
\newcommand{\E}{\mathrm{E}}
\newcommand{\Var}{\mathrm{Var}}
\newcommand{\Cov}{\mathrm{Cov}}
\newcommand{\Bias}{\mathrm{Bias}}

\begin{document}

\maketitle

\pagebreak

\begin{homeworkProblem}

    \textbf{Solution}
    \\
    \\
    \textbf{Part One}
    \[
        s(v)=1,~~\mathfrak{M},s \models \exists x(A(f(z),c) \supset \forall y (A(y,x) \lor A(f(y),x)))
    \]
    Let $\varphi_1\equiv A(f(z),c) \supset \forall y (A(y,x) \lor A(f(y),x))$ \\
    $\mathfrak{M},s \models \exists x\varphi_1$ iff $\mathfrak{M},s[m_1/x] \models \varphi_1$ for some $m_1\in\abs{\mathfrak{M}}$ \\
    $\mathfrak{M},s[m_1/x] \models A(f(z),c) \supset \forall y (A(y,x) \lor A(f(y),x))$ for some $m_1\in\abs{\mathfrak{M}}$ \\
    This is equivalent to the expression $\mathfrak{M},s[m_1/x] \not\models A(f(z),c)$ or $\mathfrak{M},s[m_1/x] \models \forall y (A(y,x) \lor A(f(y),x))$ for some $m_1\in\abs{\mathfrak{M}}$.\\
    \begin{ind}
        Evaluating $\mathfrak{M},s[m_1/x] \not\models A(f(z),c)$ for some $m_1 \in \abs{\mathfrak{M}}$: \\
        $=\langle \text{Val}^{\mathfrak{M}}_{s[m_1/x]}(f(z)), c\rangle  \not\in A^{\mathfrak{M}}$ \\
        $=\langle f^{\mathfrak{M}}(\text{Val}^{\mathfrak{M}}_{s[m_1/x]}(z)), \text{Val}^{\mathfrak{M}}_{s[m_1/x]}(c)\rangle  \not\in A^{\mathfrak{M}}$ \\
        $=\langle f^{\mathfrak{M}}(1),c^{\mathfrak{M}}\rangle  \not\in A^{\mathfrak{M}}$ \\
        $=\langle 2,3\rangle  \not\in A^{\mathfrak{M}}=\text{False}$, since $\langle 2,3\rangle $ is in $A^{\mathfrak{M}}$\\
        \\
        Evaluating $\mathfrak{M},s[m_1/x] \models \forall y (A(y,x) \lor A(f(y),x))$ for some $m_1\in\abs{\mathfrak{M}}$:\\
        Let $\varphi_2\equiv A(y,x) \lor A(f(y),x)$ \\
        $\mathfrak{M},s[m_1/x] \models \forall y\varphi_2$ iff $\mathfrak{M},s[m_1/x][m_2/y]\models\varphi_2$ for some $m_1$ for any $m_2$, such that $m_1,m_2\in\abs{\mathfrak{M}}$ \\
        This is just the expression $\mathfrak{M},s[m_1/x][m_2/y]\models A(y,x)$ or $\mathfrak{M},s[m_1/x][m_2/y]\models A(f(y),x)$, for some $m_1$ for any $m_2$. \\
        \begin{ind}
            Evaluating $\mathfrak{M},s[m_1/x][m_2/y]\models A(y,x)$, for some $m_1$ for any $m_2$: \\
            $=\langle \text{Val}^{\mathfrak{M}}_{s[m_1/x][m_2/y]}(y),\text{Val}^{\mathfrak{M}}_{s[m_1/x][m_2/y]}(x)\rangle \in A^{\mathfrak{M}}$ \\
            $=\langle m_2, m_1 \rangle \in A^\mathfrak{M}$ for some $m_1$ for any $m_2$, such that $m_1,m_2\in\abs{\mathfrak{M}}$ \\
            We must check for all possible values of $m_2$ in the domain: \\
            $=\langle 1, m_1 \rangle \in A^\mathfrak{M}$ and $\langle 2, m_1 \rangle \in A^\mathfrak{M}$ and $\langle 3, m_1 \rangle \in A^\mathfrak{M}$ \\
            \\
            Evaluating $\mathfrak{M},s[m_1/x][m_2/y]\models A(f(y),x)$, for some $m_1$ for any $m_2$. \\
            $=\langle \text{Val}^{\mathfrak{M}}_{s[m_1/x][m_2/y]}(f(y)),\text{Val}^{\mathfrak{M}}_{s[m_1/x][m_2/y]}(x) \rangle \in A^{\mathfrak{M}}$ \\
            $=\langle f^{\mathfrak{M}}(\text{Val}^{\mathfrak{M}}_{s[m_1/x][m_2/y]}(y)),m_1 \rangle \in A^{\mathfrak{M}}$ \\
            $=\langle f^{\mathfrak{M}}(m_2),m_1 \rangle \in A^{\mathfrak{M}}$ for some $m_1$ for any $m_2$, such that $m_1,m_2\in\abs{\mathfrak{M}}$ \\
            We must check for all possible values of $m_2$ in the domain: \\
            $=\langle f^{\mathfrak{M}}(1), m_1 \rangle \in A^\mathfrak{M}$ and $\langle f^{\mathfrak{M}}(2), m_1 \rangle \in A^\mathfrak{M}$ and $\langle f^{\mathfrak{M}}(3), m_1 \rangle \in A^\mathfrak{M}$ \\
            $=\langle 2, m_1 \rangle \in A^\mathfrak{M}$ and $\langle 3, m_1 \rangle \in A^\mathfrak{M}$ and $\langle 2, m_1 \rangle \in A^\mathfrak{M}$ \\
            $=\langle 2, m_1 \rangle \in A^\mathfrak{M}$ and $\langle 3, m_1 \rangle \in A^\mathfrak{M}$ \\
        \end{ind}

        $\mathfrak{M},s[m_1/x][m_2/y] \models \varphi_2$ becomes the expression:
        \[(\langle 1, m_1 \rangle \in A^\mathfrak{M} \text{ and } \langle 2, m_1 \rangle \in A^\mathfrak{M} \text{ and } \langle 3, m_1 \rangle \in A^\mathfrak{M})
            \text{ or } \\
            (\langle 2, m_1 \rangle \in A^\mathfrak{M} \text{ and } \langle 3, m_1 \rangle \in A^\mathfrak{M})
        \]
        for some $m_1 \in \abs{\mathfrak{M}}$ \\
        Let $m_1=3$, then the second half of the 'or' operation is satisfied, since $\langle2,3\rangle,\langle3,3\rangle\in A^{\mathfrak{M}}$. \\
        Therefore, $\mathfrak{M},s[m_1/x][m_2/y] \models \varphi_2$ and $\mathfrak{M},s[m_1/x] \models \forall y (A(y,x) \lor A(f(y),x))$
        
    \end{ind}
    \newpage
    Recall that $\mathfrak{M},s \models \exists x\varphi_1$ iff $(\mathfrak{M},s[m_1/x] \not\models A(f(z),c)$ or $\mathfrak{M},s[m_1/x] \models \forall y (A(y,x) \lor A(f(y),x)))$ for some $m_1\in\abs{\mathfrak{M}}$. \\
    We have shown the second half of the 'or' operation, therefore $\mathfrak{M},s \models \exists x\varphi_1$ and $\mathfrak{M},s$ satisfies the original formula.\\
    \\
    \textbf{Part Two}
    \\
    Use the originial variable assignment and structrue but have $A^{\mathfrak{M}}=\{\langle1,3\rangle\,\langle2,2\rangle\,\langle3,3\rangle\}$.

\end{homeworkProblem}

\begin{homeworkProblem}
    \textbf{Solution}
    \\

    \textbf{Part One}

    $\varphi \equiv \psi \supset \chi$ \\

    By definition, $\mathfrak{M},s_1 \models \varphi$ iff $\mathfrak{M},s_1 \not\models \psi$ or $\mathfrak{M},s_1 \models \chi$. \\
    By the IH we have $\mathfrak{M},s_2 \not\models \psi$ or $\mathfrak{M},s_2 \models \chi$. Then, $\mathfrak{M},s_2 \models \varphi$. \\

    \textbf{Part Two}
    
    $\varphi \equiv \forall x\psi$



\end{homeworkProblem}
\end{document}