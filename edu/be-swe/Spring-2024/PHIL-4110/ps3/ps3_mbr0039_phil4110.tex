\documentclass{article}

\usepackage{fancyhdr}
\usepackage{extramarks}
\usepackage{amsmath}
\usepackage{amsthm}
\usepackage{amsfonts}
\usepackage{tikz}
\usepackage[plain]{algorithm}
\usepackage{algpseudocode}
\usepackage{physics}
\usepackage{changepage}

\usetikzlibrary{automata,positioning}

%
% Basic Document Settings
%

\topmargin=-0.45in
\evensidemargin=0in
\oddsidemargin=0in
\textwidth=6.5in
\textheight=9.0in
\headsep=0.25in

\linespread{1.1}

\pagestyle{fancy}
\lhead{\hmwkAuthorName}
\chead{\hmwkClass\ (\hmwkClassInstructor\ \hmwkClassTime): \hmwkTitle}
\rhead{\firstxmark}
\lfoot{\lastxmark}
\cfoot{\thepage}

\renewcommand\headrulewidth{0.4pt}
\renewcommand\footrulewidth{0.4pt}

\setlength\parindent{0pt}

%
% Create Problem Sections
%

\newcommand{\enterProblemHeader}[1]{
    \nobreak\extramarks{}{Problem \arabic{#1} continued on next page\ldots}\nobreak{}
    \nobreak\extramarks{Problem \arabic{#1} (continued)}{Problem \arabic{#1} continued on next page\ldots}\nobreak{}
}

\newcommand{\exitProblemHeader}[1]{
    \nobreak\extramarks{Problem \arabic{#1} (continued)}{Problem \arabic{#1} continued on next page\ldots}\nobreak{}
    \stepcounter{#1}
    \nobreak\extramarks{Problem \arabic{#1}}{}\nobreak{}
}

\setcounter{secnumdepth}{0}
\newcounter{partCounter}
\newcounter{homeworkProblemCounter}
\setcounter{homeworkProblemCounter}{1}
\nobreak\extramarks{Problem \arabic{homeworkProblemCounter}}{}\nobreak{}

%
% Homework Problem Environment
%
% This environment takes an optional argument. When given, it will adjust the
% problem counter. This is useful for when the problems given for your
% assignment aren't sequential. See the last 3 problems of this template for an
% example.
%
\newenvironment{homeworkProblem}[1][-1]{
    \ifnum#1>0
        \setcounter{homeworkProblemCounter}{#1}
    \fi
    \section{Problem \arabic{homeworkProblemCounter}}
    \setcounter{partCounter}{1}
    \enterProblemHeader{homeworkProblemCounter}
}{
    \exitProblemHeader{homeworkProblemCounter}
}

\newenvironment{ind}
  {\adjustwidth{3em}{0pt}}
  {\endadjustwidth}

%
% Homework Details
%   - Title
%   - Due date
%   - Class
%   - Section/Time
%   - Instructor
%   - Author
%

\newcommand{\hmwkTitle}{Problem Set 3}
\newcommand{\hmwkDueDate}{February 26, 2024}
\newcommand{\hmwkClass}{PHIL 4110}
\newcommand{\hmwkClassTime}{}
\newcommand{\hmwkClassInstructor}{Professor Lockhart}
\newcommand{\hmwkAuthorName}{\textbf{Matthew Rogers}}

%
% Title Page
%

\title{
    \vspace{2in}
    \textmd{\textbf{\hmwkClass:\ \hmwkTitle}}\\
    \normalsize\vspace{0.1in}\small{Due\ on\ \hmwkDueDate}\\
    \vspace{0.1in}\large{\textit{\hmwkClassInstructor}}
    \vspace{3in}
}

\author{\hmwkAuthorName}
\date{}

\renewcommand{\part}[1]{\textbf{\large Part \Alph{partCounter}}\stepcounter{partCounter}\\}

%
% Various Helper Commands
%

% Useful for algorithms
\newcommand{\alg}[1]{\textsc{\bfseries \footnotesize #1}}

% For derivatives
\newcommand{\deriv}[1]{\frac{\mathrm{d}}{\mathrm{d}x} (#1)}

% For partial derivatives
\newcommand{\pderiv}[2]{\frac{\partial}{\partial #1} (#2)}

% Integral dx
\newcommand{\dx}{\mathrm{d}x}

% Alias for the Solution section header
\newcommand{\solution}{\textbf{\large Solution}}

% Probability commands: Expectation, Variance, Covariance, Bias
\newcommand{\E}{\mathrm{E}}
\newcommand{\Var}{\mathrm{Var}}
\newcommand{\Cov}{\mathrm{Cov}}
\newcommand{\Bias}{\mathrm{Bias}}

\begin{document}

\maketitle

\pagebreak

\begin{homeworkProblem}

    \textbf{Solution}
    \\
    \\
    \textbf{Part One}
    \[
        s(v)=1,~~\mathfrak{M},s \models \exists x(A(f(z),c) \supset \forall y (A(y,x) \lor A(f(y),x)))
    \]
    Let $\varphi_1\equiv A(f(z),c) \supset \forall y (A(y,x) \lor A(f(y),x))$ \\
    $\mathfrak{M},s \models \exists x\varphi_1$ iff $\mathfrak{M},s[m_1/x] \models \varphi_1$ for some $m_1\in\abs{\mathfrak{M}}$ \\
    $\mathfrak{M},s[m_1/x] \models A(f(z),c) \supset \forall y (A(y,x) \lor A(f(y),x))$ for some $m_1\in\abs{\mathfrak{M}}$ \\
    This is equivalent to the expression $\mathfrak{M},s[m_1/x] \not\models A(f(z),c)$ or $\mathfrak{M},s[m_1/x] \models \forall y (A(y,x) \lor A(f(y),x))$ for some $m_1\in\abs{\mathfrak{M}}$.\\
    \begin{ind}
        Evaluating $\mathfrak{M},s[m_1/x] \not\models A(f(z),c)$ for some $m_1 \in \abs{\mathfrak{M}}$: \\
        $=\langle \text{Val}^{\mathfrak{M}}_{s[m_1/x]}(f(z)), c\rangle  \not\in A^{\mathfrak{M}}$ \\
        $=\langle f^{\mathfrak{M}}(\text{Val}^{\mathfrak{M}}_{s[m_1/x]}(z)), \text{Val}^{\mathfrak{M}}_{s[m_1/x]}(c)\rangle  \not\in A^{\mathfrak{M}}$ \\
        $=\langle f^{\mathfrak{M}}(1),c^{\mathfrak{M}}\rangle  \not\in A^{\mathfrak{M}}$ \\
        $=\langle 2,3\rangle  \not\in A^{\mathfrak{M}}=\text{False}$, since $\langle 2,3\rangle $ is in $A^{\mathfrak{M}}$\\
        \\
        Evaluating $\mathfrak{M},s[m_1/x] \models \forall y (A(y,x) \lor A(f(y),x))$ for some $m_1\in\abs{\mathfrak{M}}$:\\
        Let $\varphi_2\equiv A(y,x) \lor A(f(y),x)$ \\
        $\mathfrak{M},s[m_1/x] \models \forall y\varphi_2$ iff $\mathfrak{M},s[m_1/x][m_2/y]\models\varphi_2$ for some $m_1$ for all $m_2$, such that $m_1,m_2\in\abs{\mathfrak{M}}$ \\
        This is just the expression $\mathfrak{M},s[m_1/x][m_2/y]\models A(y,x)$ or $\mathfrak{M},s[m_1/x][m_2/y]\models A(f(y),x)$, for some $m_1$ for all $m_2$. \\
        \begin{ind}
            Evaluating $\mathfrak{M},s[m_1/x][m_2/y]\models A(y,x)$, for some $m_1$ for all $m_2$: \\
            $=\langle \text{Val}^{\mathfrak{M}}_{s[m_1/x][m_2/y]}(y),\text{Val}^{\mathfrak{M}}_{s[m_1/x][m_2/y]}(x)\rangle \in A^{\mathfrak{M}}$ \\
            $=\langle m_2, m_1 \rangle \in A^\mathfrak{M}$ for some $m_1$ for all $m_2$, such that $m_1,m_2\in\abs{\mathfrak{M}}$ \\
            We must check for all possible values of $m_2$ in the domain: \\
            $=\langle 1, m_1 \rangle \in A^\mathfrak{M}$ and $\langle 2, m_1 \rangle \in A^\mathfrak{M}$ and $\langle 3, m_1 \rangle \in A^\mathfrak{M}$ \\
            \\
            Evaluating $\mathfrak{M},s[m_1/x][m_2/y]\models A(f(y),x)$, for some $m_1$ for all $m_2$. \\
            $=\langle \text{Val}^{\mathfrak{M}}_{s[m_1/x][m_2/y]}(f(y)),\text{Val}^{\mathfrak{M}}_{s[m_1/x][m_2/y]}(x) \rangle \in A^{\mathfrak{M}}$ \\
            $=\langle f^{\mathfrak{M}}(\text{Val}^{\mathfrak{M}}_{s[m_1/x][m_2/y]}(y)),m_1 \rangle \in A^{\mathfrak{M}}$ \\
            $=\langle f^{\mathfrak{M}}(m_2),m_1 \rangle \in A^{\mathfrak{M}}$ for some $m_1$ for all $m_2$, such that $m_1,m_2\in\abs{\mathfrak{M}}$ \\
            We must check for all possible values of $m_2$ in the domain: \\
            $=\langle f^{\mathfrak{M}}(1), m_1 \rangle \in A^\mathfrak{M}$ and $\langle f^{\mathfrak{M}}(2), m_1 \rangle \in A^\mathfrak{M}$ and $\langle f^{\mathfrak{M}}(3), m_1 \rangle \in A^\mathfrak{M}$ \\
            $=\langle 2, m_1 \rangle \in A^\mathfrak{M}$ and $\langle 3, m_1 \rangle \in A^\mathfrak{M}$ and $\langle 2, m_1 \rangle \in A^\mathfrak{M}$ \\
            $=\langle 2, m_1 \rangle \in A^\mathfrak{M}$ and $\langle 3, m_1 \rangle \in A^\mathfrak{M}$ \\
        \end{ind}

        $\mathfrak{M},s[m_1/x][m_2/y] \models \varphi_2$ becomes the expression:
        \[(\langle 1, m_1 \rangle \in A^\mathfrak{M} \text{ and } \langle 2, m_1 \rangle \in A^\mathfrak{M} \text{ and } \langle 3, m_1 \rangle \in A^\mathfrak{M})
            \text{ or } \\
            (\langle 2, m_1 \rangle \in A^\mathfrak{M} \text{ and } \langle 3, m_1 \rangle \in A^\mathfrak{M})
        \]
        for some $m_1 \in \abs{\mathfrak{M}}$ \\
        Let $m_1=3$, then the second half of the 'or' operation is satisfied, since $\langle2,3\rangle,\langle3,3\rangle\in A^{\mathfrak{M}}$. \\
        Therefore, $\mathfrak{M},s[m_1/x][m_2/y] \models \varphi_2$ and $\mathfrak{M},s[m_1/x] \models \forall y (A(y,x) \lor A(f(y),x))$
        
    \end{ind}
    \newpage
    Recall that $\mathfrak{M},s \models \exists x\varphi_1$ iff $(\mathfrak{M},s[m_1/x] \not\models A(f(z),c)$ or $\mathfrak{M},s[m_1/x] \models \forall y (A(y,x) \lor A(f(y),x)))$ for some $m_1\in\abs{\mathfrak{M}}$. \\
    We have shown the second half of the 'or' operation, therefore $\mathfrak{M},s \models \exists x\varphi_1$ and $\mathfrak{M},s$ satisfies the original formula.\\
    
    \textbf{Part Two}
    \\
    Use the originial variable assignment and structrue but have $A^{\mathfrak{M}}=\{\langle1,3\rangle\,\langle2,2\rangle\,\langle3,3\rangle\}$.

\end{homeworkProblem}

\begin{homeworkProblem}

    \textbf{Proposition 7.14.} If the free variables in $\varphi$ are among $x_1,\ldots,x_n$, and $s_1(x_i) = s_2(x_i)$ for $i = 1,\dots ,n$, then $\mathfrak{M},s_1 \models \varphi$ iff $\mathfrak{M}, s_2 \models \varphi$. \\
    Induction Hypothesis: Assume $\mathfrak{M},s_1 \models \varphi$ iff $\mathfrak{M},s_2 \models \varphi$ for all formulae $\psi$ less complex than $\varphi$. \\

    \textbf{Solution}
    \\

    \textbf{Part One}

    $\varphi \equiv \psi \supset \chi$ \\

    By definition, $\mathfrak{M},s_1 \models \varphi$ iff $\mathfrak{M},s_1 \not\models \psi$ or $\mathfrak{M},s_1 \models \chi$. \\
    By the IH we have $\mathfrak{M},s_2 \not\models \psi$ or $\mathfrak{M},s_2 \models \chi$. Then, $\mathfrak{M},s_2 \models \varphi$. \\

    The other direction of the proof relies on similar reasoning. \\

    \textbf{Part Two}

    $\varphi \equiv \forall x\psi$ \\

    If $\mathfrak{M},s_1 \models \varphi$, then for any variable assignment $s_1$, $\mathfrak{M},s_1[m/x] \models \psi$ for all $m \in \abs{\mathfrak{M}}$. \\
    Let $s_1'=s_1[m/x]$, $s_2'=s_2[m/x]$, for all $m \in \abs{\mathfrak{M}}$.\\
    By IH we get $s_1(x_i) = s_2(x_i)$ for any $x_i$ in the free variables of $\varphi$. \\
    Since $s_1',s_2'$ are x-variants of $s_1,s_2$ respectively, and the free variables of $\psi$ are in those of $\varphi$ with $x$, then we know $s_1'(x_j)=s_2'(x_j)$ for any $x_j$ in the free variables of $\psi$.\\
    Since the x-variants $s_1',s_2'$ agree on any variable assignment we can apply the IH, deriving $\mathfrak{M},s_2'\models\psi$ from $\mathfrak{M},s_1'\models\psi$. \\
    If $\mathfrak{M},s_2'\models\psi$ where $s_2'=s_2[m/x]$, for all $m \in \abs{\mathfrak{M}}$, then $\mathfrak{M},s_2 \models \varphi$. \\

    The other direction of the proof relies on similar reasoning. \\

\end{homeworkProblem}

\begin{homeworkProblem}

    \textbf{Proposition 7.17.} Let $\mathfrak{M}$ be a structure, $\varphi$ be a sentence, and $s$ a variable assignment. \\
    $\mathfrak{M} \models \varphi $ iff $\mathfrak{M},s \models \varphi$. \\

    \textbf{Proof}
    \\
    If $\mathfrak{M} \models \varphi $, then $\mathfrak{M}$ satisfies the sentence $\varphi$, and for all variable assignments $s$, $\mathfrak{M},s \models \varphi$. \\
    \\
    Working in the opposite direction, if $\mathfrak{M},s \models \varphi$, and $\varphi$ is a sentence, then corollary 7.15 allows us to say that $\mathfrak{M},s'\models\varphi$ for every variable assignment $s'$. \\
    Since $\varphi$ is a sentence and is satisfied for every $s'$, then we can write $\mathfrak{M},s\models\varphi$ for all variable assignments $s$. \\
    \\



\end{homeworkProblem}
\begin{homeworkProblem}

    \textbf{Proposition 7.18.} Suppose $\varphi(x)$ only contains $x$ free, and $\mathfrak{M}$ is a structure. Then:
    \begin{enumerate}
        \item $\mathfrak{M} \models \exists x\varphi(x)$ iff $\mathfrak{M},s \models \varphi(x)$ for at least one variable assignment $s$.
        \item $\mathfrak{M} \models \forall x\varphi(x)$ iff $\mathfrak{M},s \models \varphi(x)$ for all variable assignments $s$.
    \end{enumerate}
    \textbf{Proof} \\

    \textbf{Part One} \\
    If $\mathfrak{M} \models \exists x\varphi(x)$, then by definition $\mathfrak{M},s[m/x] \models \varphi(x)$ for some variable assignment $s$ from $x$ to some $m \in \abs{\mathfrak{M}}$. \\
    Therefore there must be at least one variable assignment $s$ such that $\mathfrak{M},s\models\varphi(x)$. \\

    Working in the opposite direction, assume $\mathfrak{M},s\models\varphi(x)$ for at least one variable assignment $s$. \\
    Since $x$ is the only free variable and $\mathfrak{M},s\models\varphi(x)$, we know $s$ must at least assign $x$ to some value $m\in\abs{\mathfrak{M}}$ such that $\varphi(x)$ is satisfied relative to $s$. \\
    Then we write $\mathfrak{M},s[m/x] \models \varphi(x)$ for some $m \in \abs{\mathfrak{M}}$, which gives way to $\mathfrak{M} \models \exists x\varphi(x)$. \\
    \\
    \textbf{Part Two} \\
    If $\mathfrak{M} \models \forall x\varphi(x)$, then by definition for any variable assignment $s$, $\mathfrak{M},s[m/x] \models \varphi(x)$ for all $m \in \abs{\mathfrak{M}}$. \\
    The satisfaction of $\varphi(x)$ relative to any $s$ is not restricted by any particular assignment of $x$, so we can write $\mathfrak{M},s\models\varphi(x)$ for all $s$. \\
    \\
    Working in the opposite direction, assume $\mathfrak{M},s\models\varphi(x)$ for all variable assignments $s$. \\
    Then any $s$ could assign $x$ to any member of the domain, and we would still have $\mathfrak{M},s\models\varphi(x)$.\\
    We can express this property of $s$ as $\mathfrak{M},s[m/x] \models \varphi(x)$ for all $m\in\abs{\mathfrak{M}}$. \\
    This implies $\mathfrak{M} \models \forall x\varphi(x)$.\\
    



\end{homeworkProblem}

\begin{homeworkProblem}

    \textbf{Proposition 7.19.} \\
    Let $\varphi$ be a formula, and $\mathfrak{M}_1$ and $\mathfrak{M}_2$ be structures with $\abs{\mathfrak{M}_1} = \abs{\mathfrak{M}_2}$, and $s$ a variable assignment on $\abs{\mathfrak{M}_1}=\abs{\mathfrak{M}_2}$.\\
    If 
    \begin{ind}
        $c^{\mathfrak{M}_1}=c^{\mathfrak{M}_2}$,\\
        $R^{\mathfrak{M}_1}=R^{\mathfrak{M}_2}$,\\
        and $f^{\mathfrak{M}_1}=f^{\mathfrak{M}_2}$\\
        for every constant symbol $c$, relation symbol $R$, and function symbol $f$ occurring in $\varphi$,
    \end{ind}    
    then $\mathfrak{M}_1,s\models\varphi$ iff $\mathfrak{M}_2,s\models\varphi$.\\

    \textbf{Proof} \\

    \textbf{Part One:} Valuation, Induction on Terms \\
    We must show that for any term $t$, $\text{Val}^{\mathfrak{M}_1}_s(t)=\text{Val}^{\mathfrak{M}_2}_s(t)$\\
    Base:
    \begin{enumerate}
        \item $t\equiv c$ \\
            $\text{Val}^{\mathfrak{M}_1}_s(c)=c^{\mathfrak{M}_1}=c^{\mathfrak{M}_2}=\text{Val}^{\mathfrak{M}_2}_s(c)$
        \item $t \equiv v$, where $v$ is a variable. \\
            $\text{Val}^{\mathfrak{M}_1}_s(v)=s(v)=\text{Val}^{\mathfrak{M}_2}_s(v)$. $s$ is the same for both domains.
    \end{enumerate}
    IH: Assume for a term $t$, for any subterm $t_i$ in $t_1\hdots t_n$,  $\text{Val}^{\mathfrak{M}_1}_s(t_i)=\text{Val}^{\mathfrak{M}_2}_s(t_i)$.
    \begin{ind}
        Let $t\equiv f(t_1\hdots t_n)$. \\
        $\text{Val}^{\mathfrak{M}_1}_s(t)=f^{\mathfrak{M}_1}(\text{Val}^{\mathfrak{M}_1}_s(t_1)\hdots\text{Val}^{\mathfrak{M}_1}_s(t_n))$ \\
        Apply the IH, and then \\
        $\text{Val}^{\mathfrak{M}_1}_s(t) = f^{\mathfrak{M}_1}(\text{Val}^{\mathfrak{M}_2}_s(t_1)\hdots\text{Val}^{\mathfrak{M}_2}_s(t_n))$ \\
        Since $f^{\mathfrak{M}_1}=f^{\mathfrak{M}_2}$ we can write \\
        $\text{Val}^{\mathfrak{M}_1}_s(t) = f^{\mathfrak{M}_2}(\text{Val}^{\mathfrak{M}_2}_s(t_1)\hdots\text{Val}^{\mathfrak{M}_2}_s(t_n))$ \\
        $\text{Val}^{\mathfrak{M}_1}_s(t) = \text{Val}^{\mathfrak{M}_2}_s(t)$ \\
    \end{ind}
    The reasoning in the other direction is symmetrical. \\

    \textbf{Part Two:} Satisfaction, Induction on Formulae \\
    We must show that for any formula $\varphi$, $\mathfrak{M}_1,s\models\varphi$ iff  $\mathfrak{M}_2,s\models\varphi$ \\
    Base: $\varphi$ is atomic
    \begin{enumerate}
        \item $\varphi\equiv \bot$ \\
            By definition, any $\mathfrak{M},s$ satisfies $\bot$, therefore $\mathfrak{M}_1,s\models\varphi$ and $\mathfrak{M}_2,s\models\varphi$
        
        \item $\varphi \equiv R(t_1\hdots t_n)$ \\
            $\mathfrak{M}_1,s\models\varphi$ iff $\langle\text{Val}^{\mathfrak{M}_1}_s(t_1)\hdots\text{Val}^{\mathfrak{M}_1}_s(t_n)\rangle\in R^{\mathfrak{M}_1}$ \\
            From our valuation proof, we know $\text{Val}^{\mathfrak{M}_1}_s(t_1)\hdots\text{Val}^{\mathfrak{M}_1}_s(t_n) = \text{Val}^{\mathfrak{M}_2}_s(t_1)\hdots\text{Val}^{\mathfrak{M}_2}_s(t_n)$. Rewrite: \\
            $\langle\text{Val}^{\mathfrak{M}_2}_s(t_1)\hdots\text{Val}^{\mathfrak{M}_2}_s(t_n)\rangle\in R^{\mathfrak{M}_1}$ \\
            Since $R^{\mathfrak{M}_1}=R^{\mathfrak{M}_2}$, \\
            $\langle\text{Val}^{\mathfrak{M}_2}_s(t_1)\hdots\text{Val}^{\mathfrak{M}_2}_s(t_n)\rangle\in R^{\mathfrak{M}_2}$ \\
            Then $\mathfrak{M}_2,s\models\varphi$

        \item $\varphi \equiv =(t_1,t_2)$ \\
            $\mathfrak{M}_1,s\models\varphi$ iff $\text{Val}^{\mathfrak{M}_1}_s(t_1)=\text{Val}^{\mathfrak{M}_1}_s(t_2)$\\
            We know $\text{Val}^{\mathfrak{M}_1}_s(t)=\text{Val}^{\mathfrak{M}_2}_s(t)$, so\\
            $\text{Val}^{\mathfrak{M}_2}_s(t_1)=\text{Val}^{\mathfrak{M}_2}_s(t_2)$ \\
            Then $\mathfrak{M}_2,s\models\varphi$
    \end{enumerate}
    IH: Assume the result holds for all formulae less complex than $\varphi$.
    \begin{enumerate}
        \item $\varphi \equiv \sim\psi$ \\
            $\mathfrak{M}_1,s\models\varphi$ iff $\mathfrak{M}_1,s\not\models\psi$ \\
            If $\mathfrak{M}_1,s\not\models\psi$, then by IH $\mathfrak{M}_2,s\not\models\psi$ \\
            Then $\mathfrak{M}_2,s\models\varphi$

        \item $\varphi \equiv \psi \land \chi$ \\
            $\mathfrak{M}_1,s\models\varphi$ iff $\mathfrak{M}_1,s\models\psi$ and $\mathfrak{M}_1,s\models\chi$ \\
            Then by IH $\mathfrak{M}_2,s\models\psi$ and $\mathfrak{M}_2,s\models\chi$ \\
            Then $\mathfrak{M}_2,s\models\varphi$

        \item $\varphi \equiv \psi \lor \chi$ \\
            $\mathfrak{M}_1,s\models\varphi$ iff $\mathfrak{M}_1,s\models\psi$ or $\mathfrak{M}_1,s\models\chi$ \\
            Then by IH $\mathfrak{M}_2,s\models\psi$ or $\mathfrak{M}_2,s\models\chi$ \\
            Then $\mathfrak{M}_2,s\models\varphi$

        \item $\varphi \equiv \psi \supset \chi$ \\
            $\mathfrak{M}_1,s\models\varphi$ iff $\mathfrak{M}_1,s\not\models\psi$ or $\mathfrak{M}_1,s\models\chi$ \\
            Then by IH $\mathfrak{M}_2,s\not\models\psi$ or $\mathfrak{M}_2,s\models\chi$ \\
            Then $\mathfrak{M}_2,s\models\varphi$

        \item $\varphi \equiv \exists x\psi$ \\
            $\mathfrak{M}_1,s\models\varphi$ iff $\mathfrak{M}_1,s[m/x]\models\psi$ for some $m\in\abs{\mathfrak{M}_1}$ \\
            Since $s[m/x](x)=m\in\abs{\mathfrak{M}_1}$, and $\abs{\mathfrak{M}_1}=\abs{\mathfrak{M}_2}$, then with IH we can write: \\
            $\mathfrak{M}_2,s[m/x]\models\psi$ for some $m\in\abs{\mathfrak{M}_2}$ \\
            Then $\mathfrak{M}_2,s\models\varphi$

        \item $\varphi \equiv \forall x\psi$ \\
            $\mathfrak{M}_1,s\models\varphi$ iff $\mathfrak{M}_1,s[m/x]\models\psi$ for all $m\in\abs{\mathfrak{M}_1}$ \\
            Since $s[m/x](x)=m\in\abs{\mathfrak{M}_1}$, and $\abs{\mathfrak{M}_1}=\abs{\mathfrak{M}_2}$, then with IH we can write: \\
            $\mathfrak{M}_2,s[m/x]\models\psi$ for all $m\in\abs{\mathfrak{M}_2}$ \\
            Then $\mathfrak{M}_2,s\models\varphi$
    \end{enumerate}
    The reasoning in the other direction is symmetrical. \\

    By induction, we've shown if two structures declare the same domain, constants, functions, relations, and variable assignments, then they must only satisfy the same formulae.






\end{homeworkProblem}
\end{document}